%% LyX 2.1.4 created this file.  For more info, see http://www.lyx.org/.
%% Do not edit unless you really know what you are doing.
\documentclass[12pt,english]{article}
\usepackage[T1]{fontenc}
\usepackage[latin9]{inputenc}
\usepackage[a4paper]{geometry}
\geometry{verbose,tmargin=1.25cm,bmargin=1.25cm,lmargin=1.25cm,rmargin=1.25cm}
\usepackage{babel}
\usepackage[unicode=true]
 {hyperref}

\makeatletter
%%%%%%%%%%%%%%%%%%%%%%%%%%%%%% User specified LaTeX commands.
\usepackage{tikz}
\usetikzlibrary{arrows}

\makeatother

\begin{document}

\title{DSA Assignment 1:Well Formed Polygons}


\author{Joe Kurian Eappen\\
EE13B080}

\maketitle

\section{Introduction}

Program was done in C++ with the use of the STL. Vectors were used
to facilitate dynamically allocated structures ( such as number of
sides in a Polygon, number of Polygons,etc.). This could have been
done with preallocated memory spaces with the limiting cases in mind
as well.


\subsection{Data Structures And Special Variables Used}
\begin{enumerate}
\item Constants

\begin{enumerate}
\item Constants used in the Adjacency Matrix : \\
const int NOT\_CHECKED=-2; const int NOT\_CONNECTED=-1; \\
const int INTERSECTING=1; const int ITSELF=0;
\item Defining the bounds of our coordinate space:\\
const int END\_OF\_X=32000; const int END\_OF\_Y=32000;
\item Maximum distance between any two polyons in the graph. Used in place
of INF in the shortest path algo. \\
const float MAX\_DIST=1000000;
\end{enumerate}
\item Structures:

\begin{enumerate}
\item Point: To store X and Y coordinates of a point where both are non
negative integers.
\item Side: Consisting of two Point Structures.

\begin{enumerate}
\item Preprocessing was done to store the left coordinate of a particular
side in the first Point structure and the right coordinate in the
second for all edges.
\end{enumerate}
\item CentroidPoint: To store the X and Y coordinates of the Centroid of
a Polygon ( float used since it may have decimal values)
\item edge: To store the 'to' node and length of the edge in the AdjacencyList
representation.
\end{enumerate}
\item Classes

\begin{enumerate}
\item Polygon:

\begin{enumerate}
\item Members:

\begin{enumerate}
\item vectors x,y : to store the i-th coordinates in x{[}i{]} and y{[}i{]}.
In hndsight, a vector of the Point struct would have been a better
choice but this was implemented later.
\item bool convexity : true if shape was convex, false if not (intended
to implement faster algorithms for convex polygons but spent time
on other aspects of the program)
\item Vector of Side struct

\begin{enumerate}
\item sides : Sides in the same order as inputed
\item sidesOrderedByX: Sides ordered in ascending order of the leftmost
coordinate
\end{enumerate}
\item CentroidPoint cent : To store the centroid of the polygon
\item int xmax,xmin,ymax,ymin: To store the values needed to construct the
bounding rectangle of the Polygon. 
\end{enumerate}
\item Important Member Functions:

\begin{enumerate}
\item bool CheckIfAntiClockwise():

\begin{enumerate}
\item Checks if point given are in the anticlockwise order.
\item Uses the fact that the integral of the area is negative if going in
the anticlockwise direction.
\end{enumerate}
\item bool CheckConvexity():

\begin{enumerate}
\item Check if polygon is convex using the fact that any 3 points in order
should have clockwise orientation if the polygon is convex.
\end{enumerate}
\item int orientation(Point p, Point q, Point r):

\begin{enumerate}
\item Returns 0,1 or 2 depending on if p,q, and r are collinear, anticlockwise
or clockwise (in our unusual coordinate system where -y axis is +y.)
\item Used the slopes of the line from Point p to q and q to r to make the
decision.
\item Algo provided \href{http://www.geeksforgeeks.org/orientation-3-ordered-points/}{here}.
\end{enumerate}
\item bool onSegment(Point p, Point q, Point r):

\begin{enumerate}
\item Returns true if Point q is on the line segment connecting p and r.
\end{enumerate}
\item bool doIntersect(Point p1, Point q1, Point p2, Point q2, bool DoingPointInPolyCheck=false):

\begin{enumerate}
\item if DoingPointInPolyCheck=false then Check if line p1q1 and p2q2 are
intersecting excluding the case that one point lies on the other line
(or is collinear)
\item else consider that case as well.
\item This was needed to include the condition that touching edges or a
vertice on an edge was not considered to be intersecting ( which was
a major reason for most hiccups in the code and a significant part
of the problem statement.)
\end{enumerate}
\item bool CheckIfIntersecting():

\begin{enumerate}
\item To check if any of the edges of the polygon intersects with the other.
\item This was made before the idea of using an Edge structure or a vector
of such structures came to mind and iterates through all points constructing
edges as we move along.
\end{enumerate}
\item bool CheckPointInPolygon(Point p):

\begin{enumerate}
\item Checks if a given pont P is inside the polygon or not (includes edges).
\item First checks if Point is outside the bounding rectange of the polygon.
\item If it is inside, counts number of intersections of a line y=(x coordinate
of P). If this is even then the point is outside, else it is inside.
\item If collinear then check if point lies on the collinear edge else don't
count it.
\end{enumerate}
\item bool CheckPointStrictlyInPolygon(Point p):

\begin{enumerate}
\item Checks if a given pont P is strictly inside the polygon or not.
\end{enumerate}
\item Constructor function taking vector<int> x,vector<int> y:

\begin{enumerate}
\item Calculated bounding rectangle, centroid, convexity and populates sides
and sidesOrderedByX.
\end{enumerate}
\end{enumerate}
\end{enumerate}
\item SetOfPolygons:

\begin{enumerate}
\item Members:

\begin{enumerate}
\item vector<Polygon> PolyList : a vector of the Polygons in the program
\item vector<float> distances: The distances to each polygon from the first
one (if graph is connected).
\item vector< vector<int> > AdjacencyMatrix: Matrix used to check for intersections.
-2 for not checked, -1 for infinity, 1 for Intersecting (see constants).
\item vector< vector<edge> > AdjacencyList : Adjacency List representation
of th graph used in the shortest path algorithm.
\item int DisconnectedPoly, bool Disconnected :\\
In case the graph is disconnected, store the position of the disconnected
polygon.\end{enumerate}
\end{enumerate}
\end{enumerate}
\end{enumerate}

\end{document}
